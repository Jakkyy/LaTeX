\documentclass[]{article}

\usepackage[paperheight=18cm,paperwidth=14cm,textwidth=12cm]{geometry}
\usepackage[skip=20pt plus1pt, indent=40pt]{parskip}

\usepackage{hyperref}

\usepackage{graphicx}
\graphicspath{ {./images/} }
\usepackage{float}

\usepackage{amsmath}
\usepackage{amsfonts}
\usepackage{amssymb}

\usepackage{xcolor}
\usepackage{sectsty}
\definecolor{bittersweet}{rgb}{1.0, 0.44, 0.37}
\definecolor{grey}{rgb}{0.25, 0.25, 0.28}
\definecolor{black}{rgb}{0, 0, 0}
\subsubsectionfont{\color{black}}
\subsubsectionfont{\color{grey}}
\sectionfont{\color{bittersweet}}

\usepackage[T1]{fontenc}
\renewcommand\familydefault{\sfdefault} 

\usepackage{bm}

\newcommand{\ev}{\mathbb{E}[X]}
\renewcommand{\ev}[1]{\mathbb{E}[#1]}

\newcommand{\definizione}{\paragraph{Definizione:}}
\newcommand{\formula}{\paragraph{Formula generica:}}

\newcommand{\highlight}[1]{\colorbox{yellow}{$\displaystyle #1$}}

\begin{document}
    \tableofcontents
    \newpage
    
    \section{Introduzione}
    $X_1 = 1.7$ \\
    $X_2 = 1.82$ \\
    $X_3 = 1.73$\\
    $X_4 = 1.7$\\
    $X_5 = 1.8$ \\
    $\hat{\theta}? \quad \text{Altezza della popolazione}$
    \paragraph{Possibile soluzione}: 
    \[ \hat{\theta_a} = \frac{1}{n} \sum_{4}^{5} x_i = \frac{1.7 + 1.82 + 1.73 + 1.7 + 1.8}{5} = \frac{8.75}{5} = 1.75 \]
    \[ \hat{\theta_b} = \frac{min(x_i) + max(x_i)}{2} = \frac{3.52}{2} = 1.76 \]
    \[ \hat{\theta_c} = \frac{1}{3} \sum_{2}^{4} x_i = \frac{1}{3} (1.8 + 1.73 + 1.7) = \frac{5.23}{3} = 1.743 \]
    \centerline{Scartiamo il più \textit{piccolo} e il \textit{massimo}, calcolando poi la \textbf{media} dei rimanenti}
    \paragraph{Stima parametrica}(Point) Parametric Estimation \\ \\
    \underline{Ipotesi}:
    - Esiste un parametro $\theta$ incognito $n$ dati a disposizione $\{X_1, X_2, X_n\}$ \\
    \textbf{Legge di probabilità} che descrive il fenomeno che ha generato i dati
    \formula Bayes
    \[ P(\theta / X_1 \ldots X_n) = \frac{P(X_1 \ldots X_n / \theta) P(\theta)}{P(X_1 \ldots X_n)} \]
    \centerline{Verosomiglianza (likelihood)}
    \paragraph{MLE} Maximum Likelihood Estimation (Stima a Massima Verosomiglianza)
    \[ \hat{\theta} = argmax L(\theta) = argmax[f(X_1 \ldots X_n / \theta )] \]
    \paragraph{Esempio} (Legge -> Distribuzione di Poisson)
    \begin{equation*}
        \begin{split}
            f(X_1, X_2 \ldots X_n / \theta) &= f(X_1 / \theta) \cdot f(X_2 / \theta) \ldots f(X_n / \theta) \\
            &= \frac{1}{\theta} e^{-\frac{X_1}{\theta}} \cdot \frac{1}{\theta} e^{-\frac{X_2}{\theta}} \cdot \ldots \frac{1}{\theta} e^{-\frac{X_n}{\theta}} \\
            &= \frac{1}{\theta n} e^{-\frac{1}{\theta} \sum_{i}^{} X_i }
        \end{split}
    \end{equation*}
    \paragraph{Esempio} (MLE Ipotesi di Bernoulli)
    \begin{equation*}
        X_i =
        \begin{cases}
            0 \\
            1
        \end{cases}
    \end{equation*}
    \[ P\{X_i = 1\} = 1 - P\{X_i = 0\} \]
    \[ P\{X_i = x\} = P^x (1-P)^x \quad x \in \{0, 1\} \]
    \centerline{Dove \textbf{X} è una \textit{variabile aleatoria} e \textbf{x} una \textit{variabile sperimentale} }
    \[ f(x_1 \ldots x_n / P) = P^{x_1} (1-P)^{1-x_1} \cdot P^{x_2} (1-P)^{1-x_2} \ldots P^{x_n} (1-P)^{1-x_n} = \]
    \[ P^{\sum_{i}^{n}x_i} (1-P)^{n - \sum_{1}^{n} x_i} \longrightarrow \text{Bisogna trovare il \textbf{massimo} della funzione} \]
    \begin{equation*}
        \begin{split}
            log(f(x_1 \ldots x_n / P)) &= \sum_{1}^{n} x_i log P - (n - \sum_{i}^{n} x_i) log(1-P) \\
            &= \frac{d}{dP}[log(f)] = 0 = \frac{1}{\hat{P}} \sum_{i}^{n} x_i - \frac{n - \sum_{i}^{} x_i }{(1-\hat{P})} \\
            &= (1- \hat{P}) \sum_{i}^{} x_i = \hat{P} (n - \sum_{i}^{} x_i) \\
            &= \hat{P} = \frac{\sum_{i}^{} x_i }{n} \quad \text{MLE}
        \end{split}
    \end{equation*}
    \paragraph{Esercizio 1} Probabilità che Oneto dia 30L (Lode) \\
    $n = 120$ \\
    $\sum_{i}^{120} x_i = 18$ \\
    $\hat{P} = \frac{18}{120} = 0.15 \rightarrow 15\%$ \\
    \paragraph{Esercizio 2} N studenti da 30 e lode \\
    $n_1 = 18 \leftarrow \text{Oneto}$ \\ 
    $n_2 = 20 \leftarrow \text{Anguita}$ \\
    $n_{1,2} = 10 \leftarrow \text{30L sia con Oneto che con Anguita}$ \\
    $N= ? \quad \text{Studenti da \textbf{30 e Lode}}$ \\
    \begin{minipage}{0.30\textwidth}
        \[ \hat{P_1} \approx \frac{n_1 2}{n_2} \]
    \end{minipage}
    \begin{minipage}{0.30\textwidth}
        \[ \hat{P_1} \approx \frac{n_1}{N} \]
    \end{minipage}
    \begin{minipage}{0.30\textwidth}
        \[ \frac{n_{1,2}}{n_2} = \frac{n_1}{N} \]
    \end{minipage} \\
    $ \Longrightarrow N= \frac{n_1 \cdot n_2}{n_{1,2}} \rightarrow \frac{18 \cdot 20}{10} = 36 $
    \paragraph{MLE POISSON}
    \begin{equation*}
        \begin{split}
            f(x_1, x_2 \ldots x_n / \lambda) &= \frac{e^{-\lambda} \lambda^{x_1}}{x_1 !} \cdot \frac{e^{-\lambda} \lambda^{x_2}}{x_2 !} \cdots \frac{e^{-\lambda} \lambda^{x_n}}{x_n !} \\
            &= \frac{e^{-n\lambda} \lambda^{\sum_{i}^{} x_i}}{x_1! x_2! \ldots x_n!}
        \end{split}
    \end{equation*}
    \formula
    $\lambda = \frac{\sum_{i}^{} x_i}{\lambda} \quad \text{MLE} $
    \paragraph{Esercizio 3} Stima del numero di incidenti medio in auto n = 10 \\
    $x_1 = \{ 4,0,6,5,2,1,2,0,4,3 \}$ \\
    $\hat{\lambda} = \frac{\sum_{i}^{} x_i}{n} = \frac{27}{10} = 2.7$
    \[ P\{x \leq 2 \} = e^{-2.7} (\frac{2.7^0}{0!} + \frac{2.7^1}{1!} + \frac{2.7^2}{2!}) \approx .4936 \rightarrow 49.36\% \]
    \centerline{Probabilità che non ci siano più di \textbf{2 incidenti} }
    \paragraph{MLE UNIFORME}
    \begin{equation*}
        f(x_1, x_2 \ldots x_n / \theta) =
        \begin{cases}
                \frac{1}{\theta^n} & \quad 0 < x_i < \theta \\
                0 & \quad \text{altrimenti}
        \end{cases}
    \end{equation*}
    $\hat{\theta} = max\{x_i\}$ \\
    $\frac{\hat{\theta}}{2} = \frac{max\{ x_i\}}{2}$
    \paragraph{MLE GAUSSIANA}
    \[ f(x_1,x_2 \ldots x_n / \mu, \sigma) = \prod_{i=1}^{n} \frac{1}{\sqrt{2\pi} \sigma} e^{\frac{-(x_1 - \mu)^2}{2 \sigma^2}} \]
    \[ (\frac{1}{2 \pi})^{\frac{n}{2}} \frac{1}{\sigma^n} e^{\frac{-\sum_{i}^{}(x_i - \mu)^2}{2 \sigma}} \]
    $log[f] = - \frac{n}{2} log 2\pi - n log \sigma - \frac{\sum_{i}^{}(x_i - \mu)^2}{2\sigma^2}$ \\ \\
    $\frac{d log f}{d \mu} = 0 = \frac{\sum_{i}^{}(x_i - \mu)^2}{\sigma^2} \longrightarrow \hat{\mu} = \frac{\sum_{i}^{}x_i}{n}$ \\
    $\frac{d log f}{d \sigma} = 0 = - \frac{n}{\sigma} + \frac{\sum_{i}^{} (x_i - \mu)^2}{4 \sigma^4} \rightarrow \sigma = \sqrt{\frac{\sum_{i}^{}(x_i - \mu)^2}{n}}$
    \paragraph{Esercizio} primo \\
    $x_1 = 1.7$ \\
    $x_2 = 1.82$ \\
    $x_3 = 1.73$\\
    $x_4 = 1.7$\\
    $x_5 = 1.8$ \\
    \[ \hat{\mu} = \frac{\sum_{i}^{} x_i}{n} = \frac{1.7 + 1.82 + 1.73 + 1.7 + 1.8}{5} = 1.75 \]
    \[ \hat{\sigma} = \sqrt{\frac{0.05^2 + 0.07^2 + 0.02^2 + 0.05^2 + 0.05^2}{5}} \approx 0.051 \]
    \paragraph{Intervalli di confidenza} normali
    TODO
    \paragraph{Intervalli di confidenza} gaussiani
    $\sigma^2$ Nota \\
    $x_1m x_2 \ldots x_n$ \\
    $\hat{\mu} \longleftarrow \mu$ \\
    $\frac{\overline{x} - \mu}{\frac{\sigma}{\sqrt{n}}} \sim \mathcal{N}(0,1)$ \\
    \[ P(-1.96 < \frac{\overline{x} - \mu}{\frac{\sigma}{\sqrt{n}}} < +1.96) = 0.95 \]
    \[ \longrightarrow P(-1.96 \frac{\sigma}{\sqrt{n}} < \overline{x} - \mu < 1.96 \frac{\sigma}{\sqrt{n}}) \]
    \[ P(\overline{x} - 1.96 \frac{\sigma}{\sqrt{n}} < \mu < \overline{x} + 1.96 \frac{\sigma}{\sqrt{n}}) \]
    \paragraph{Esempio:} Sistema di comunicazione
    $\sigma^2 = 4 \quad n = 9$
    \[ x_1 = \{ 5.85, 12, 15, 7, 9, 7.5, 6, 5, 10.5 \} \]
    \[ \hat{\mu} = \frac{1}{n} \sum_{i}^{n} x_i = \frac{1}{9} \sum_{i}^{n} x_i = \frac{81}{9} = 9 \] \\
    \begin{equation*}
        \begin{aligned}
            & P\left(9-1.96 \frac{\sigma}{\sqrt{m}}<\mu<9+1.96 \frac{\sigma}{\sqrt{m}}\right)=0.95 \\
            & p\left(9-1.96 \frac{2}{3}<\mu<9+1.96 \frac{2}{3}\right)=0.95 \\
            & \longrightarrow[7.693,10.31] \rightarrow \mu \text { si trova tra 7.693 e 10.31} \\
        \end{aligned}
    \end{equation*}
    \paragraph{In generale} Prob = $1-\alpha$
    \[ (\overline{x} - z_a \frac{\sigma}{\sqrt{n}}, \overline{x} + z_a \frac{\sigma}{\sqrt{n}} ) \rightarrow \text{Si rileva dalle tavole} \]
    \subsection{Intervalli di confidenza (Bilaterali)}
    \[ \overline{X} = \frac{1}{n} \sum_{i}^{n} x_i \]
    \[ x_i \sim \mathcal{N}(\mu, \sigma^2) \]
    \[ \overline{X} \sim(\mu, \frac{\sigma^2}{n}) \]
    \[ \mathcal{Z} = \frac{\overline{X} - \mu}{\frac{\sigma}{\sqrt{n}}} \sim \mathcal{N}(0, 1) \quad Var(\frac{x}{2}) = \frac{1}{\sigma^2} Var(x) \]
    \text{Supponiamo che $\sigma$ sia nota:}
    \begin{equation*}
        \begin{aligned}
            &\begin{aligned}
            & \operatorname{Pr}\left\{-z_{\frac{\alpha}{2}}<Z<+z_{\frac{\alpha}{2}}\right\}=1-\alpha \\
            & \operatorname{Pr}\left\{-z_{\frac{\alpha}{2}}<\frac{\bar{x}-\mu}{\frac{\sigma}{\sqrt{m}}}<+z_{\frac{\alpha}{2}}\right\}=1-\alpha \\
                & \operatorname{Pr}\left\{-z_{\frac{\alpha}{2}} \frac{\sigma}{\sqrt{m}}<\bar{x}-\mu<+z_{\frac{\alpha}{2}} \frac{\sigma}{\sqrt{m}}\right\}
        \end{aligned}\\
        &\begin{aligned}
        & \operatorname{Pr}\left\{-\bar{x}-z_{\frac{\alpha}{2}} \frac{\sigma}{\sqrt{m}}<-\mu<-\bar{x}+z_{\frac{\alpha}{2}} \frac{\sigma}{\sqrt{m}}\right\}= \\
        & \operatorname{Pr}\left\{\bar{x}-z_{\frac{\alpha}{2}} \frac{\sigma}{\sqrt{m}}<\mu<\bar{x}+z_{\frac{\alpha}{2}} \frac{\sigma}{\sqrt{m}}\right\}=1-\alpha
        \end{aligned}
        \end{aligned}
    \end{equation*}
    \subsection{Intervalli di confidenza (Unilaterali)}
    \begin{equation*}
        \begin{aligned}
        & \operatorname{Pr}\left\{z<z_\alpha\right\}=1-\alpha \\
        & \operatorname{Pr}\left\{\frac{\bar{x}-\mu}{\frac{\sigma}{\sqrt{m}}}<z_\alpha\right\}=1-\alpha \\
        & \operatorname{Pr}_r\left\{\bar{x}-\mu<z_\alpha \frac{\sigma}{\sqrt{m}}\right\}=1-\alpha \\
        & \operatorname{Pr}\left\{-\mu<-\bar{x}+z_\alpha \frac{\sigma}{\sqrt{m}}\right\}=1-\alpha \\
        & \operatorname{Pr}\left\{\bar{x}-z_\alpha \frac{\sigma}{\sqrt{m}}<\mu\right\}=1-\alpha \\
        & \mu \in\left(\bar{x}-z_\alpha \frac{\sigma}{\sqrt{m}},+\infty\right)
        \end{aligned}
    \end{equation*}
    \subsection{Esempio:} Pesca stagionale dei salmoni (\textit{Fisso intervallo -> trovo $n$}) \\
    \text{Ad ogni stagione il peso medio dei salmoni è diverso ma $\sigma = 0.3$ Kg} \\
    \text{Intervallo di confidenza al 95\%, quindi $\alpha = 0.05$}
    \[ (\overline{X} - 1.96 \frac{\sigma}{\sqrt{n}}, \overline{X} + 1.96 \frac{\sigma}{\sqrt{n}}) \]
    \[ 1.96 \frac{\sigma}{\sqrt{n}} \geq 0.1 \quad \sqrt{n} \geq \frac{1.96}{0.1} \sigma \]
    \[ n \geq (\frac{1.96}{0.1} 0.3)^2 = 5.88^2 \approx 34.6 \leftarrow \text{salmoni} \]
    \subsection{Intervallo di confidenza} con \textit{media} e \textit{varianza} \textbf{incognite}
    \begin{equation*}
        \begin{aligned}
            & Z=\frac{\bar{x}-\mu}{\frac{\sigma}{\sqrt{n}}} \sigma \quad \text{ Non nota} \\
            & s^2=\frac{1}{n-1} \sum_i\left(x_i-\bar{x}\right)^2=\frac{1}{n-1} \sum_i^n\left(x_i^2-n \bar{x}^2\right) \\
            & =\frac{1}{n-1} \sum_i\left(x_i^2+\bar{x}^2-2 x_i \bar{x}\right) \\
            & =\frac{1}{n-1} \sum_i x_i^2+\frac{n \bar{x}^2}{n-1}-2 \bar{x} \frac{\bar{x} n}{n-1} \\
        \end{aligned}
    \end{equation*}
    \[ T = \frac{\overline{X} - \mu}{\frac{s}{\sqrt{n}}} \sim T_n - 1 \quad \text{(T studenti con n gradi di libertà)} \]
    \paragraph{Esempio:} Trasimttente ($\mu$) e ricevitore ($\mu$ + rumore)
    \[ 95\% (7.69, 10.31) \quad \hat{\mu} = 9, \sigma^2 = 4 \]
    $X_i \{ 5, 8.5, 12, 15, 7, 9, 7.5, 6.5, 10.5 \}$ \\
    $\hat{\mu} = \overline{X} = \frac{1}{9} \sum_{i}^{n} X_i = \frac{81}{9} = 9$ \\
    $s^2 = \frac{1}{8}\sum_{i}^{}(X_i^2-9.81) \approx 9.5 \quad s = 3.082$
    \[ \mu \in (9-2.306 \frac{3.082}{3}, 9 + 2.306 \frac{3.082}{3}) = (6.63, 11.37) \]
    \centerline{Si può dimostrare che $T_{\frac{\alpha}{2} \cdot n - 1} \ev{S} \geq z_\alpha\sigma$}
    \subsection{Integrali Monte Carlo}
    \[ \theta = \ev{f(u)} = \int_{-\infty}^{+\infty} f(u) p(u) \, du = \int_{-\infty}^{+\infty} f(u) \, du \]
    \paragraph{Esempio}: \\
    $\int_{0}^{1} \sqrt{1-x^2} \,dx = ?$
    $\ev{\sqrt{1-x^2}} \quad n = 100$ \\
    $X_i = \sqrt{1 - U_i^2} \quad X = \{ X_1, X_2 \ldots X_100 \}$ \\
    $\hat{\theta} = \overline{X} \pm t_{\frac{\alpha}{2}}, 99 \frac{s}{\sqrt{100}} \rightarrow \text{Per vedere se il risultato è corretto \textit{(confidenza)}}$ 
    \subsection{Intervallo di confidenza di Bernoulli}
    n esperimenti \\
    Binomiale \\
    media np \\
    varianza np(1-p) \\
    \[ \hat{P} = \frac{1}{n} \sum_{i}^{n}X_i \quad X_i \in \{ 0, 1 \} \] \\
    \[ X = n\hat{P} \quad P_r \{ -z_{\frac{\alpha}{2}} < z < z_{\frac{\alpha}{2}} \approx 1-\alpha \}\] \\
    \centerline{Dove z = $\frac{X-np}{\sqrt{np(1-p)}}$}
    \[ \frac{x-nP}{\sqrt{nP(1-P)}} \sim \mathcal{N}(0,1)\]
    \begin{equation}
        \begin{aligned}
        & \rho_r\left\{-z_{\frac{a}{2}}<\frac{x-m p}{\sqrt{m p(1-\hat{p})}}<z_{\frac{a}{2}}\right\} \cong 1-\alpha \\
        & \rho_r\left\{\hat{p}-z_{\frac{a}{2}} \sqrt{\frac{p(1-p)}{m}}<\mu<\hat{p}+z_{\frac{\alpha}{2}} \sqrt{\frac{\hat{p}(1-p)}{m}}\right\} \simeq 1-\alpha
        \end{aligned}
    \end{equation}
\end{document}